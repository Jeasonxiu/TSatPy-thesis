\chapter{Misc. Ramblings}

Like any piece of software \LaTeX has its shares of quirks and little
odds and ends. Here I will share a few notes on lists and finally add
some comments on some of the commands that I have found to be useful.

\section{Lists}

I have already given an example of one kind of list using the
itemize environment.  A complete list is given below \cite{The-Manual}:
\begin{enumerate}
\item The itemize list.
\item An enumerate list.
\item A description list.
\end{enumerate}
The different types of lists are described below 
\cite[p. 65]{The-Manual}:
\begin{description}
\item[itemize]     Lists each item with a $\bullet$.
\item[enumerate]   Gives a numbered list.
\item[description] Used to describe the members of a list.
\end{description}


You can also put a list inside a list.  If you use
enumerate it will even keep track of the numbers for you:
\begin{enumerate}
\item Top Five Reasons to take Calculus:
    \begin{enumerate}
     \item Annoy roommate when pulling all-nighters.
     \item At 8:00AM it's too dark for volleyball.
     \item Three words: The chain rule.
     \item Can learn how to use $\Sigma$\  without 
           referring to alcohol.
     \item Can impress date with knowledge of the limit.
     \end{enumerate}
\item Top Five Worst Reasons to take Calculus:
     \begin{enumerate}
     \item Would like to learn how to use that spiffy calculator.
     \item If you have to spend a lot of money on a book it should
             at least be a really thick book, damnit.
     \item Would like to learn new French and German names to
             impress people at parties.
     \item Satisfies the liberal arts school's requirement
           for ``class that requires that you know how to  count''.
     \item Morning class is motivation to go get breakfast 
           before they run out of those pancakes.
     \end{enumerate}
\end{enumerate}


\section{Window Dressings}
Since some of you may actually like having a typesetting
program around, I thought I might like to share some
of the little things that help round out the \lat\ program.
For example, I often use \lat\ for letters.  When I want
to write one of my colleagues, Einar Ronqui\o st, I need
to know how to use some extra characters.  Other non-English
characters include the following: \oe, \OE, \ae,
\AE, \aa, \AA, \o, \O, \l, \L, \ss,
?`, !`.

People from foreign lands also like to use accents which
are not included on the standard keyboard.  To get around
this \lat\ has some commands for \`{a}ccents.  See Table
\ref{accents} for a list \cite[p. 40 (blatant rip-off)]{The-Manual}
and Table \ref{mathaccents} for a list of accents for use
in math mode \cite[p. 51]{The-Manual}.

\newcommand{\bs}{$\backslash$}
\begin{table}[hbpt]
\begin{tabular}{r @{-} l @{\hspace{0.5in}} r @{-} l  @{\hspace{0.5in}}
                r @{-} l @{\hspace{0.5in}} r @{-} l }
\`{a} & \bs `\{a\} & 
\~{a} & \bs \~\{a\} & 
\v{a} & \bs v\{a\} & 
\c{o} & \bs c\{o\}  \\
\'{a} & \bs '\{a\} & 
\={a} & \bs =\{a\} & 
\H{a} & \bs H\{a\} & 
\d{o} & \bs d\{o\}  \\
\^{a} & \bs \^\{a\} & 
\.{a} & \bs .\{a\} & 
\t{aa} & \bs t\{aa\} & 
\b{o} & \bs b\{o\}  \\
\"{a} & \bs "\{a\} & 
\u{a} & \bs u\{a\} 
\end{tabular}
\caption{Some accents for you to use.}
\label{accents}
\end{table}

\begin{table}[hbpt]
\begin{tabular}{r @{-} l @{\hspace{0.5in}} r @{-} l  @{\hspace{0.5in}}
                r @{-} l @{\hspace{0.5in}} r @{-} l }
$\hat{u}$   & \bs hat\{u\} & 
$\acute{u}$ & \bs acute\{u\} & 
$\bar{u}$   & \bs bar\{u\} & 
$\dot{u}$   & \bs dot\{u\}  \\
$\check{u}$ & \bs check\{u\} & 
$\grave{u}$ & \bs grave\{u\} & 
$\vec{u}$   & \bs vec\{u\} & 
$\ddot{u}$  & \bs ddot\{u\}  \\
$\breve{u}$ & \bs breve\{u\} & 
$\tilde{u}$ & \bs tilde\{u\}
\end{tabular}
\caption{Some accents for you to use in math mode.}
\label{mathaccents}
\end{table}


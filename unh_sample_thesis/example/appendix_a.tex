\chapter{Equations}

Some examples of how to insert mathematical expressions is given in
this appendix. The first set of examples focuses on equations. The
second set of examples provides some details on how to work with
arrays. Finally, some examples are given to show how to work with
different symbols.


\section{Equations}
There are two modes in \LaTeX\ that we will use.  The first is text
mode.  This is the default mode and is used to simply typeset this
text.  The second is math mode.  You must explicitly tell \LaTeX\ when
you want to use math mode.  There are two ways to do this.  If you
want to put symbols in with your text you must use math mode, and you
do this by starting math mode using a \$ and ending with another \$.
Symbols such as the $\backslash$\ must be done in math mode.  At the
end of math mode or a slash command I use a slash followed by a space.
This is done to insure that the proper spacing is used by \LaTeX.  I
do not use a slash space when the symbol is followed by punctuation.


% In the next paragraph I will introduce how to start an
% equation environment.  I will also give the proper credit
% to the source.  I will make a citation using the \cite
% command.  The citation that I use is defined at the end
% of this document in the bibliography.  To use a citation
% you must define the label.  Because the definition is at
% the end of the file the first time you use latex you will
% have to run it through twice to make sure the definitions
% are updated.  (This is my one real gripe about latex)

The other way to start math mode is in an equation environment.  There
are different ways to do this but the most common is to use eqnarray
\cite{The-Manual}.  This environment is started and ended in the same way
the document environment is employed.  For example to give the formula
for a straight line I might do it in slope-intercept format
\cite[p. 8]{thomas},
\begin{eqnarray}
y & = & mx + b.
\end{eqnarray}
The \LaTeX\ program will automatically give the equation a number
and center it as well.  Note that I used two ampersands to set off the
$=$\ sign.  This is done to let \LaTeX\ know where to center the
equation \cite{The-Manual}.  This way I can put more equations in the
equation array and keep them properly justified \cite[p. 7]{thomas},
\begin{eqnarray}
y & = & mx + b, \\
y - y_0 & = & m (x - x_0).
\end{eqnarray}
In some situations, I may have a large number of equations but
only want to see one number.  To keep \LaTeX\ from printing out
a number each time use the nonumber command \cite{The-Manual},
\begin{eqnarray}
y       & = & mx + b \\
y - y_0 & = & m (x - x_0). \nonumber
\end{eqnarray}
(The \LaTeX\ program will automatically typeset things for you so 
you can make your life much easier if you line up things in the
text file.  This will make editing much easier.)

I have introduced something here without telling you.  The \LaTeX\ 
program will let you make subscripts and superscripts using the 
carrot and lower bar symbols.  If I want to make a subscript or superscript
with just one letter it can be done as in the equations above.
If I want to make subscripts and superscripts with more letters I
have to let \LaTeX\ know which letters are to be used through the
use of the braces \cite{The-Manual},
\begin{eqnarray}
y_{approximate} & = & m^{guess} x_{known} + b^{guess}.
\end{eqnarray}
When you know these basics you can go on to create great documents.


Another very useful tool is the new command.  Since I am getting
tired of typing out \LaTeX\ whenever I want the Latex symbol
to appear I can define a new command to do it for me.  In this
case I will call this new command $\backslash$lat.
\newcommand{\lat}{\LaTeX} 
Now whenever I want to use the \lat\ symbol life is much easier.


\section{The Array Environment}
Sometimes the mathematical formulas you have are a little more complicated
than the equation for a line.  The equation array environment defines
three columns and will justify the  text in those columns.  For 
example we can have a simple equation,
\begin{eqnarray}
y & = & x^2.
\end{eqnarray}
In this example, the first column contains the $y$, the second column
contains the $=$, and the third column contains the $x^2$.  The \lat\
program will automatically right justify the first column, center the
second column, and left justify the third column \cite{The-Manual}.
There are times when you would like to have more than three columns.
For example, I would like to put a bound on the values for $x$.  Here
an equation is constructed which has four columns: the first is right
justified, the second is centered, the third is left justified, and
the fourth is left justified.  I will use the array environment which
will be nested {\it inside} the equation array environment
\cite{The-Manual},
\begin{eqnarray}
    \begin{array}{rcll}
       y & = & x^2 & x \in [-1,1].
    \end{array}
\end{eqnarray}
After I defined the array, I had to tell \lat\ how many columns
and how they are to be justified.  Once done, I had to tell
\lat\ how the text was to be put in the columns.  The ampersand
symbol is used to distinguish between columns \cite{The-Manual}.

This same method is used to make arrays of numbers.  For example
a matrix is a simple block of numbers (what we do with them is not
quite as simple),
\begin{eqnarray}
    \begin{array}{rrrr}
       1 &  2 &  3 &  4 \\
      11 & 12 & 13 & 14 \\
      21 & 22 & 23 & 24 \\
      31 & 32 & 33 & 34.
    \end{array}
\end{eqnarray}
This looks kind of messy so I would like to spruce things up.  I want
to put braces on each side of the array.  To do this I have to let
\lat\ know where to put the symbol and which symbol to use.  This is
done through the use of the left and right commands \cite{The-Manual},
\begin{eqnarray}
\left[
    \begin{array}{rrrr}
       1 &  2 &  3 &  4 \\
      11 & 12 & 13 & 14 \\
      21 & 22 & 23 & 24 \\
      31 & 32 & 33 & 34
    \end{array}
\right].
\end{eqnarray}
Here I put the left and right commands {\it outside} of the
array.  The character following the command is the character
to be used and they do not have to be the same \cite{The-Manual},
\begin{eqnarray}
\left\{
    \begin{array}{rrrr}
       1 &  2 &  3 &  4 \\
      11 & 12 & 13 & 14 \\
      21 & 22 & 23 & 24 \\
      31 & 32 & 33 & 34
    \end{array}
\right).
\end{eqnarray}
If I do not want a closing character on the right side I must still
let \lat\ know where to close (this will let me do more complicated
things with nested brackets and such).  If there will be no symbol on
one side use a period to let \lat\ know where to end the current
set of delimiters \cite{The-Manual},
\begin{eqnarray}
\left\{
    \begin{array}{rrrr}
       1 &  2 &  3 &  4 \\
      11 & 12 & 13 & 14 \\
      21 & 22 & 23 & 24 \\
      31 & 32 & 33 & 34.
    \end{array}
\right. 
\end{eqnarray}

Using the array environment we can now place an array inside 
one of the columns of another array.  The most common example
of this would be the use of an array in one of the standard
columns of the equation array.  For example, we can now define
$y$\ as a function of $x$\ where the formula to be used depends
on $x$,
\begin{eqnarray}
y & = & \left\{
           \begin{array}{lrcccl}
              x^2           & -1 & < & x & \leq & 0 \\
              \sin(\pi x)   &  0 & < & x & <    & 1.
           \end{array}
        \right.
\end{eqnarray}


\section{Mathematical Symbols}
Other than the low cost of the program, one of the nice things
about the \lat\ program is the ease in which symbols can be 
used in formulas and mixed with text.  In this section I will
try to give a list of some of the symbols you will be using.
First, the commands for the Greek letters are given.  Next,
the commands for mathematical functions are given, and then
various other important symbols are given. This is not a 
complete list but is merely some of the symbols that are most
often used.  For the complete list consult any of the \lat\ 
manuals in the bookstore.

\subsection{Greek Letters}
The \lat\ program will let you display all of the Greek letters
in both lower and upper case.  The lower case letters are displayed
using all lower case letters \cite{The-Manual},
\begin{eqnarray}
    \begin{array}{ccccccc}
       \alpha & \beta     & \gamma    & \delta   & \epsilon & \varepsilon & \zeta \\
       \eta   & \theta    & \vartheta & \iota    & \kappa   & \lambda     & \mu   \\
       \nu    & \xi       & o         & \pi      & \varpi   & \rho        & \varrho \\
       \sigma & \varsigma & \tau      & \upsilon & \phi     & \varphi     & \chi   \\
       \psi   & \omega
    \end{array}
\nonumber
\end{eqnarray}
To use an upper case Greek letter use the same command only the 
first letter should be upper case.  For example, $\Pi$\ is the
upper case version of $\pi$ \cite{The-Manual}.

\subsection{Math Functions}
When printing text in math mode \lat\ uses italic type.  This 
can be a problem when you want to use a function in a formula,
\begin{eqnarray}
y & = & sin(\pi x).
\end{eqnarray}
The letters in the function sine are in italic, and it is hard to tell
the difference between the letters used elsewhere.  To make it easier
to differentiate between the different things in your formulas, \lat\
makes it easy for you to use a different typeset.  The most common
functions that are used have their own \lat\ commands.  Going back to
our example, we can let the reader know the difference between the
variables and the functions,
\begin{eqnarray}
y & = & \sin( \pi x).
\end{eqnarray}
(I used a space between the command to make $\pi$\ and the $x$.
When in math mode \lat\ will print things out without the spaces.)

Other functions that have \lat\ commands are listed below \cite{The-Manual},
\begin{eqnarray}
   \begin{array}{lllllll}
      \arccos & \arcsin & \arctan & \cos    & \cosh   & \cot & \coth \\
      \csc    & \exp    & \lim    & \liminf & \limsup & \ln  & \log  \\
      \max    & \min    & \sec    & \sin    & \sinh   & \sup & \tan \\
      \tanh
   \end{array}
\end{eqnarray}

\subsection{Math Symbols}
Besides the Greek letters and the various math functions, the \lat\ 
program has many symbols.  Here a very brief list of these symbols
is given \cite{The-Manual}.  For a more complete list please see one of the \lat\ 
books in the bookstore.

\begin{eqnarray}
   \begin{array}{lllllll}
      \pm     & \times  & \div      & \cap    & \cup   & \triangle & \leq \\
      \subset & \in     & \geq      & \supset & \equiv & \approx   & \neq \\
      \propto & \perp   & \parallel & \smile  & \frown & \imath    & \jmath \\
      \Re     & \Im     & \prime    & \nabla  & \angle & \forall   & \infty \\
      \sum    & \int    & \partial
   \end{array}
\end{eqnarray}




\subsection{Putting it Together}
Once you know how to put different symbols in your document the \lat\ 
program has a few tools that will let you combine things and display
different operations.  For example, \lat\ will let you write things
as a fraction.  This is done with the frac command which takes two
arguments \cite{The-Manual}.  The first argument is the numerator and the second operator
is the denominator.  The different arguments are set apart using braces,
\begin{eqnarray}
   y & = & \frac{\sin(x)}{x}.
\end{eqnarray}

Other symbols may require more information than just the symbol.  For
example, when using the sum symbol, ($\sum$), you have to let your
reader know where to start adding and when to stop.  Since these
arguments are found above and below the symbols they are added through
the use of superscripts and subscripts,
\begin{eqnarray}
   \sum^{N}_{i=0}  \left( \frac{1}{s} \right)^i
             & = & \frac{1 - \left( \frac{1}{s}\right)^{N+1}}
                        {1-\frac{1}{s}}
\end{eqnarray}
As you can see, some of the formulas you use require you to use
operations that are nested within operations.  The formulas you use
can become quite complicated in short order.  When you type these
formulas in it will be to your advantage to do so carefully and to
make them readable.  Otherwise, you may be in for some long nights
in the computer center!

Another symbol that you will use is the integral symbol,
\begin{eqnarray}
\int        x dx & = & \frac{1}{2} x^2 + c \\
\int^1_{-1} x dx & = & 0. \nonumber
\end{eqnarray}
The limits are added using superscripts and subscripts.
One problem with the formulas above is that \lat\ likes
to run everything together in math mode.  When you want a
space in a formula you must explicitly tell \lat\ where
to put the space.  This is done with the same command
used while in text mode \cite{The-Manual},
\begin{eqnarray}
\int^1_{-1} x \ dx & = & 0.
\end{eqnarray}


Another function that may prove useful is the square
root function \cite{The-Manual},
\begin{eqnarray}
\sqrt{x^3} & = & x^{\frac{3}{2}}. 
\end{eqnarray}

Finally, another useful thing is to put one symbol
over another without the bar you see in fractions.
This is done using stackrel \cite{The-Manual}. 
$\stackrel{\circ \circ}{\smile}$

\subsection{Remembering Equation Numbers}
When you have equations in your document you will most likely want to
refer to the equations in the text.  Rather than trying to figure out
what each equation number is \lat\ will let you put a label on an
equation and refer to the label \cite{The-Manual}.  This way, when you
add and delete equations as your document ferments you will not have
to go back and change equation numbers.  I will give some examples of
how this is done and then will let you know the dangers.

First, to refer to an equation you must first put a label on the
equation,
\begin{eqnarray}
\label{derivative}
f'(x) & = & \frac{d}{dx} \left( x \sin(x) \right).
\end{eqnarray}
Once labeled, I can easily refer to equation (\ref{derivative}).
To make the reference look like the original equation I had
to put the parenthesis around the reference myself.  It is
considered bad form if your reference does not look like the
original.  

There are some warnings that should be heeded.  The way that \lat\
numbers things can be a little quirky.  The \lat\ program keeps a
running tab of the current equation number.  When it meets an
equation that does not have a nonumber command it assigns the current
number to the equations and {\em then} increments the equation number
\cite{The-Manual}.  So if I have an equation with two lines I have to put
the label in the right place,
\begin{eqnarray}
\label{thereallabel}
\frac{d}{dx} f(x) & = &  x \sin(x), \\
f(x)  & = & -x \cos(x) + \sin(x).
\label{antiderivative}
\nonumber
\end{eqnarray}
Now when I want to say something about (\ref{antiderivative})
I might not get the right number! However, if I refer to
equation (\ref{thereallabel}) I will get the correct number.

There is one other thing that you should keep in mind.  The \lat\
program is a bit primitive in the way that it remembers things.  The
program must keep tabs of the equation numbers so that when you refer
to them it can use the equation number.  Because you might want to
refer to an equation before it is given, the \lat\ program must play
some book-keeping tricks.  To do this the program creates a file {\em
after} you run your document through the program.  Then the {\em next}
time you use \lat\ it can figure out what equation numbers to use.
When you make changes that involve the numbering of the equations you
may have to run your document through \lat\ twice to get it right.  No
one ever said that life would be easy $\ldots$

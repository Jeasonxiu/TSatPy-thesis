
\pagebreak
\chapter*{Abstract}
\addcontentsline{toc}{chapter}{Abstract}
\label{chap:abstract}

\begin{center}
DEVELOPMENT OF A HIGH-INTEGRITY ANALYTICAL AND EXPERIMENTAL TESTBED FOR OBSERVER-BASED CONTROLLERS FOR NASA's MAGNETOSPHERIC MULTISCALE MISSION SPACECRAFT \\
by\\
Daniel Robert Couture\\
University of New Hampshire, May, 2014\\
\vspace{0.4cm}
\end{center}

\TODO{Pull highlight from completed thesis}
% This thesis presents several methods for the on-board and/or ground-based calibration of accelerometers for the spacecraft (s/c) of the NASA Magnetospheric MultiScale (MMS) Mission during mission operation. A lumped bias is estimated to correct for the total effect of the MMS accelerometer sensor bias, orthogonal misalignment and the shift in the s/c's center of mass.

% Various estimation techniques are evaluated and compared, including both dynamically driven real-time filters/observers and post processing batch algorithms. Both methods are shown to accurately determine lumped bias, so long as the s/c inertia tensor is well known. If, however, there is any uncertainty in the inertia tensor, only post processing methods yield accurate lumped bias estimates. Analytical simulations show that these methods are able to correct accelerometer readings to within 1 micro-g of true acceleration. Preliminary experimental verification also shows proof of concept.



%%%%% ^^ 136 words currently (14 more allowable)
%
%This thesis will present multiple (methods) for the calibration of the accelerometers used on-board NASA's MMS spacecrafts. A lumped bias will be estimated to (simultaneously) account for the accelerometer's sensor bias, orthogonal misalignment as well as a shift in the spacecraft's center of mass. The lumped bias will be shown to effectively correct for all unknown parameters to within 1 micro-g of the true (state/acceleration).
%
% Various methods will be (compared/evaluated) and both dynamically driven real-time filters as well as post-processing batch algorithms will be shown to successfully determine the lumped bias if the inertia tensor is known accurately. If there is error in the (estimated/measured) inertia tensor, however only the post processing method yielded an (accurate) lumped bias (term/vector).

%No more than 150 words, no tables, no figures, no symbols now sub/superscripts or greek letter.


%what it is
%what its for/why
%what the research will show


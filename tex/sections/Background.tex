
\chapter{BACKGROUND}
\label{chap:Background}

The National Aeronautics and Space Administration (NASA) is planning to launch the Magnetospheric Multiscale (MMS) Mission into orbit around the Earth in order to study reconnection of the magnetic fields.  The effects of the phenomenon can be clearly seen in disturbances to spacecraft, GPS signals, and even inducing current in the power grid to the point of causing outages.  Although the effects are readily observed the internal dynamics and modeling is only theorized.  The causes of magnetic field reconnection trace back to coronal mass ejections (CME) from the sun.  These CMEs contain their own magnetic field properties that on collision with the Earth's fields can cause opposing field lines to cross and then separate.  This crossing and separation is called reconnection.  Its occurrences can open up portals allowing plasma from the interstellar medium to connect with plasma of the Earth's magnetosphere and produce a release of a large amount of energy.  A reconnection event starts on the dayside magnetopause of the Earth and
rapidly folds over to magnetotail.  The diffusion region of the event can travel at about \TODO{lookup speed} and has a thickness of \TODO{lookup thickness} which can pass orbiting satellites in a fraction of a second.  Theories about the internal structure of the diffusion region \TODO{reference active theories} have been difficult to test because of the speed of the event.  The general approach to getting a full view of the area surrounding a satellite is to mount a sensor facing out from the satellite and set the satellite to be spin stabilized so that through the turn a series of observations can be made and then stitched together to form a full field of view.  In order for this approach to be partially successful the satellite would have to be spinning at about 600 rpm to get a single full view image, which for a moderate sized satellite would introduce a restrictive amount of rotational dynamics coumpounded by the fact that each MMS satellite has 4 Spin-plane Double Probe (SDP) booms \TODO{work in
reference to UNH R. B. Torbert for FIELDS investigation} extending about 50 meters out from the spacecraft (s/c) body and two Axial Double Probe (ADP) booms extending 10 meters out along the rotational axis.  The potential 600 rpm speed would translate into a SDP tip velocity around 3,000 meters per second.
The approach taken by the MMS team is to mount a series of instruments around the s/c to have a full view at all times.  This in conjunction with the rate of measurements each instrument can take will provide a much richer set of data for the mission scientists to analyze.


\section{Related Work}

UNH Work

Dissertations and THeses @ University of New Hampshire:
http://search.proquest.com.libproxy.unh.edu/pqdtlocal1006039/advanced?accountid=14612

\subsection{A Comparative Analsysis of Body-Rate Estimation Techniques for the NASA Magnetospheric Multiscale (MMS) Mission Spacecraft}

Mushaweh, N. and Jenkins, B. and Castelli, D. and Thein, M. L., "A Comparative Analsysis of Body-Rate Estimation Techniques for the NASA Magnetospheric Multiscale (MMS) Mission Spacecraft," AIAA-2009-5949, Proceedings of the 2009 AIAA Guidance, Navigation, and Control Congress and Exposition, Chicago, IL, August 2009

\subsection{Comparative Observer-Based Nutation Control Techniques for NASA Magnetospheric Multiscale (MMS) Mission Spacecraft}

Mushaweh, N. and Thein, M. L., "Comparative Observer-Based Nutation Control Techniques for NASA Magnetospheric Multiscale (MMS) Mission Spacecraft," AIAA-2008-7484, Proceedings of the 2008 AIAA Guidance, Navigation, and Control Congress and Exposition, Honolulu, HI, August 2008.

\subsection{TableSat Generation II: a Limited 5-DOF Table Top Prototype for NASA's Magnetospheric MultiScale (MMS) Mission}

Waterhouse, D. and Dunstan, M. and Kramer, M. and Kite, J. and Jenkins, B. and Thein, M. L., "TableSat Generation II: a Limited 5-DOF Table Top Prototype for NASA's Magnetospheric MultiScale (MMS) Mission," accepted for publication in the Proceedings of the 2010 AIAA/AAS Astrodynamics Specialist Conference, August 2010.

\subsection{TableSatII for NASA's Magnetospheric MultiScale (MMS) Mission - A Problem in Orbit and Attitude Determination and Control}

Waterhouse, D. and Dunstan, M. and Kramer, M. and Kite, J. and Jenkins, B. and Thein, M. L., "TableSatII for NASA's Magnetospheric MultiScale (MMS) Mission - A Problem in Orbit and Attitude Determination and Control, " AAS 10 - 255,Proceedings of the 20th AAS/AIAA Space Flight Mechanics Meeting, San Diego, CA, February 2010

\subsection{Feedback Control During Orbital Transfers of a NASA MMS Spin-Stabilized Spacecraft}

Borrelli, M. J. and Thein, M. L., "Feedback Control During Orbital Transfers of a NASA MMS Spin-Stabilized Spacecraft," AIAA-2008-7084, Proceedings of the 2008 AIAA/AAS Astrodynamics Specialist Congress and Exposition, Honolulu, HI, August 2008.

\subsection{Hybrid Modeling Technique Applied to NASA's Magnetospheric MultiScale (MMS) Mission TableSat Generation IB (TableSat IB)}

Jason Medas, University of New Hampshire, Durham; May-Win Thein, University of New Hampshire, Durham
Chapter DOI: 10.2514/6.2012-4870
Publication Date: 13 August 2012 - 16 August 2012

\subsection{Non - UNH}

\subsection{SYSTEM MODELING AND CONTROLLER
DESIGN FOR A SINGLE DEGREE
OF FREEDOM SPACECRAFT SIMULATOR}

Missy

Control systems theory is an important field of study for many branches of
engineering. Teaching control systems to engineering students, however, is often
difficult due to the abstract nature of the subject. TableSat is a single degree of
freedom spacecraft simulator that includes sensors, actuators, a power system,
and a flight processor. Students can use TableSat to design and test controllers,
allowing them to see how theoretically designed controllers function in a real
system. TableSat, like all real systems, is highly nonlinear. To make TableSat an
effective teaching tool, the system nonlinearities are identified and compensation
methods undertaken to eliminate those nonlinearities. Linear and truth system
models are created for use in controller design and testing. The system models
are tested and verified and then used to design and test several controllers and estimators. Results are presented that compare results for the linear model, truth
model, and real TableSat system.

\subsection{The TableSat Platform and its Verifiable Control Software}

The TableSat single degree-of-freedom tabletop satellite platform was developed to support
education and research activities in embedded software and control systems. This paper describes
the TableSat system, actuated by a pair of computer fans as thrusters, and then focuses on the
evolution of its software for sensor calibration and fault-tolerant control. Sensor systems include a
three-axis magnetometer, core sun sensor suite, and single-axis rate gyro. The sun sensor and
magnetometer together provide redundancy in pointing direction measurement, supporting limited
sensor fusion and enabling the system to continue operation given redundant sensor failure
situations. Baseline software was developed to support rate control and pointing control based
either on magnetometer or sun sensor data. This software was encapsulated within the ObjectBench
code generation system to support formal software specification. Guarantees of execution properties
such as sequencing and timing constraints were established with the ObjectCheck model checking
system. Design choices, evaluation processes, and results from all phases of software development,
testing and verification are presented. Use of the TableSat system as an interdisciplinary research
and education tool is also discussed.


\subsection{Incorporating Resource Safety Verification to Executable Model-based
Development for Embedded Systems}

This paper formulates and illustrates the integration of
resource safety verification into a design methodology for
development of verified and robust real-time embedded
systems. Resource-related concerns are not closely linked
with current xUML model-based software development
although they are critical for embedded systems. We
describe how to integrate resource analysis techniques
into the early phase of an xUML-based development cycle.
Our hybrid framework for resource safety verification
combines static resource analysis and runtime monitoring.
A case study based on an embedded controller for
satellite simulation, TableSat, illustrates the benefits
obtained by incorporating resource verification into
design and combining static analysis and runtime
monitoring.

\subsection{Mission-Aware Cyber-Physical Optimization on a Tabletop Satellite}


Justin M. Bradley, University of Michigan, Ann Arbor; Meghan Clark, University of Michigan, Ann Arbor; Ella M. Atkins, University of Michigan, Ann Arbor; Kang G. Shin, University of Michigan, Ann Arbor
Chapter DOI: 10.2514/6.2013-4807
Publication Date: August 19-22, 2013

http://arc.aiaa.org/doi/abs/10.2514/6.2013-4807

\subsection{ObjectCheck: A Model
Checking Tool for Executable Object-Oriented Software System
Designs}

http://web.cecs.pdx.edu/~xie/pubs/ObjectCheck.pdf

\subsection{ROBUST OBSERVER BASED RELIABLE CONTROL FOR SATELLITE
ATTITUDE CONTROL SYSTEMS WITH SENSOR FAULTS}

The problem of both robustness to parameter uncertainty and fault-tolerant
to sensor faults for satellite attitude control system is discussed in this paper. More
general and practical continuous type of sensor fault is considered here. Both prior known
sensor faults and unknown sensor faults are investigated for satellite attitude control
systems. Reliable controller based on a modified observer is realized by means of LMI,
which guarantees the stability of the closed-loop system and reduces the conservatives of
the system design. Numerical simulation demonstrates the effectiveness of the proposed
method.

\subsection{Comparative study of various control methods for attitude control of a LEO satellite}

A nonlinear attitude model of a satellite with thrusters, gravity torquers and a reaction wheel cluster is developed. Then the linearized version of this satellite attitude model is derived for the attitude hold mode. For comparison purposes, various control methods are considered for attitude control of a satellite. We consider a proportional derivative controller which is actually used in the remote sensing satellite, KOMPSAT. Then a comparison is made with an H2 controller, an H∞ controller, and a mixed View the MathML source controller. The analysis and numerical studies show that the proportional derivative controller's performance is limited in the sense that the pitch angle cannot approach zero. The simulations also show that among three control methods (H2 control, H$\infty$ control, and mixed H2/H$\infty$ control) H2 control has the fastest response time, H$\infty$ control has the slowest and mixed H2/H$\infty$ control has an intermediate value. On the other hand, H$\infty$ control used least amount of control effort while H2 control required the most.


\subsection{Design
of
a Set of Reaction Wheels for Satellite Attitude Control Simulation}

At
Instituto
Tecnológico de Aeronáutica, we are working in conception, design and development of a tri-
dimensional attitude control experiment. It consists of an aluminum hollow sphere, that floats on a micro-metrical air
layer supported by an aerostatic bearing. The sphere, denominated here Satellite Attitude Simulator (SAS), can turn
freely without friction, reproducing in this way, in laboratory, the satellite in its orbit in the space. The sphere contains
the sensors, in this case the three-axis gyroscopes, and the actuators, consists of three orthogonally installed reaction
wheels. An external computer system communicates with the mentioned devices to perform an attitude control action.
This work reports the design, mathematical modeling and parameters definition for the reaction wheels, used as actuators
in this Satellite Attitude Simulator. The first issue addressed in this work is the definition of the minimum requirements for
the three actuators of the SAS system. Then, the reaction wheels are dimensioned, and with their design parameters the
mathematical modeling is performed, and by iteratively repeated simulation of the mathematical model with change of
parameters, the selection of the reaction wheels is optimized. With this iterative simulation, the goal is to test actuators of
different dimensions, that met the design requirements of the SAS, but also minimize its weight and energy consumption.
This mathematical model is also used to evaluate the performance of control algorithms that make use of the sensors as
inputs and generates actuation signal as outputs to activate the reaction wheels. It will be effective also for discussions
about the treatment of disturbances caused by nonlinearities in the reaction wheels. The simulations results obtained with
the mathematical model are compared with experimental data, to validate the model


\subsection{Attitude Control of a Satellite Simulator Using Reaction Wheels and a PID Controller}

Attitude requirements of a satellite are determined by its mission: telecommunications, optical imagery, and meteorology to name a few. A satellite's ability to orient its mission critical hardware (solar arrays, attitude sensors, etc.), as well as its mission specific payload, is incumbent upon the performance of the satellite's attitude control system (ACS). For a highly accurate ACS and for moderately fast maneuverability, reaction wheels are preferred because they allow continuous and smooth control while inducing the smallest possible disturbance torques. The objective of this research is to design, build, test, and evaluate the performance of a reaction wheel ACS on-board the Air Force Institute of Technology's (AFIT) second generation satellite simulator, SimSat II. The reaction wheel ACS is evaluated against performance measures set forth by AFIT faculty; specifically, the ability to perform rest-to-rest maneuvers and withstand worst case disturbance torques. In all, the reaction wheel ACS proves it is capable of performing rest-to-rest maneuvers and withstanding disturbance torques. However, results conclude that theoretical predicted performance is unattainable. The performance of the reaction wheel ACS hinges upon its ability to command the reaction wheels at fixed interval timing. The inability of the test bed to execute fixed interval timing caused performance degradation.

\subsection{Compare Tabletop style platform with umbrella style platform of air bearings spacecraft simulator}

The “Tabletop” and “umbrella” style platforms of 3DOF air-bearing spacecraft simulator were compared. According to the theory on kineto-elastodynamic analysis and on the spacecraft attitude kinetics, the formulary for the platform structure mass-center displacement is deduced. It is calculated that the unbalance torque resulted from deforming of the platform of air-bearing simulator acted on gravity. Both of the styles can be compared by analyzing unbalance torque influence by changing of distance between the plate and the center of rotation of the simulator.The result shows that tabletop style platform has such advantages as lower equilibrium sensitivity, higher adaptability for payload than umbrella style platform. However, because the rotation center of air bearing of tabletop style platform locates the plate of the platform, the maximal deflection angle of the simulator was restricted.

\subsection{Attitude Control Actuators for a Simulated Spacecraft}

Christopher McChesney, Air Force Institute of Technology
Chapter DOI: 10.2514/6.2011-6509
Publication Date: 08 August 2011 - 11 August 2011

http://arc.aiaa.org/doi/abs/10.2514/6.2011-6509

\subsection{SIMSAT - A satellite simulator and experimental test bed for Air Force research}

James Colebank, USAF, Inst. of Technology, Wright-Patterson AFB, OH; Robert Jones, USAF, Inst. of Technology, Wright-Patterson AFB, OH; George Nagy, USAF, Inst. of Technology, Wright-Patterson AFB, OH; Randall Pollak, USAF, Inst. of Technology, Wright-Patterson AFB, OH; Donald Mannebach, USAF, Inst. of Technology, Wright-Patterson AFB, OH; Stuart Kramer, USAF, Inst. of Technology, Wright-Patterson AFB, OH; Gregory Agnes, USAF, Inst. of Technology, Wright-Patterson AFB, OH
Chapter DOI: 10.2514/6.1999-4428
Publication Date: 28 September 1999 - 30 September 1999

http://arc.aiaa.org/doi/abs/10.2514/6.1999-4428

SIMSAT (the SIMulation SATellite) is an AFIT-sponsored program to develop a laboratory-based physical satellite simulator. SIMSAT supports experimentation in areas of attitude control, precision pointing, and vibration suppression. SIMSAT development began with the purchase of a three-axis gas bearing which simulates a torque-free space environment. SIMSAT subsystems provide power, attitude control, telemetry, and structural support to experimental payloads. Project challenges included integration of attitude control software/hardware, high-output power storage devices, remote communications, high-frequency data collection, structural design, and computer display outputs.


\subsection{SIMSAT: An object oriented architecture for real-time satellite simulation (March 1, 1993)}

Real-time satellite simulators are vital tools in the support of satellite missions. They are used in the testing of ground control systems, the training of operators, the validation of operational procedures, and the development of contingency plans. The simulators must provide high-fidelity modeling of the satellite, which requires detailed system information, much of which is not available until relatively near launch. The short time-scales and resulting high productivity required of such simulator developments culminates in the need for a reusable infrastructure which can be used as a basis for each simulator. This paper describes a major new simulation infrastructure package, the Software Infrastructure for Modelling Satellites (SIMSAT). It outlines the object oriented design methodology used, describes the resulting design, and discusses the advantages and disadvantages experienced in applying the methodology.


\subsection{Attitude Control and Multimedia Representation of Air Force Institute of Technology's (AFIT's) Simulation Satellite (SIMSAT)}

This document describes the systematic construction of the AFIT-sponsored program to develop a laboratory-based satellite simulator. The simulation satellite (SIMSAT) system will provide a useful tool for resident staff while teaching attitude control concepts. A brief overview of attitude control theory is provided as well as a discussion of the benefits of multimedia use in education. A detailed discussion of the satellite's components allows the reader to become familiar with each piece of SIMSAT. Software control models are provided as well as a multimedia lesson plan on satellite attitude control. Also included in this document are potential experimental uses in the areas of attitude control, precision pointing, and vibration suppression as well as continued modification of the multimedia presentation capabilities.

\subsection{Design and Testing of SIMSAT, a three-axis satellite dynamics simulator}

Gregory Agnes, USAF, Inst. of Technology, Wright-Patterson AFB, OH; Joseph Fulton, U.S. Air Force Academy, Colorado Springs, CO
Chapter DOI: 10.2514/6.2001-1591
Publication Date: 11 June 2001 - 14 June 2001


Read More: http://arc.aiaa.org/doi/abs/10.2514/6.2001-1591Gregory Agnes, USAF, Inst. of Technology, Wright-Patterson AFB, OH; Joseph Fulton, U.S. Air Force Academy, Colorado Springs, CO
Chapter DOI: 10.2514/6.2001-1591
Publication Date: 11 June 2001 - 14 June 2001

Read More: http://arc.aiaa.org/doi/abs/10.2514/6.2001-1591


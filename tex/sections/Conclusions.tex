\chapter{CONCLUSIONS}
\label{chap:Conclusions}

The goal of this thesis was to determine the viability of the NASA MMS TableSat 1A's platform for comparison and experimental verification of various observer-based controllers.


Prior implementations of TableSat generally stuck with a static actuator layout which requires a significant portion of time to rewrite code and/or rewire electronics to assess variations on the actuator layout.  With the use of TSatPy, most modifications to the actuators during testing can be done with just a change to the config structure that provides a much faster turn around time between tests.


\TODO{Finish this}

Estimator:
  SMO bettor for initial response
  Toss up between SMO/PID for ss response, SMO update state sooner, PID less error

Controller:
  SMC rate controller performs far better with perfect estimates

State Error
  Multiplicative quaternion corrections
  Scaling based or the represented angle rather than the quaternion scalar
  Estimating the vector quantities of the quaternion can interfere with the angular component, only control the use of the angular component (fewer parameters to tune)

Python so much better than Numerical Simulation Software (like Matlab or Octave) for programming
Slow progress to establish the foundation, but became faster as building blocks were created and vetted through unit tests

run-time feedback ++

Attitude control
  Surprising level of control with just a single gain in a P-controller!!!


Antipodal response with P-attitude and body rate control

Quaternion decomposition Equation \ref{eqn:DecomposeQuaternion}



TSatPy

This module is designed to run a variety of observer based controls with a rotating rigid body.

Notable characteristics of this implementation of a rotating body's attitude observer based controller are:

Adaptive Step Algorithms

All time dependent calculations vary their parameters at run-time dependent on the time since the last time it ran. For example, calculations based on integrals or derivatives reference the system clock to scale the results based on what the clock reports is the current time step.

Variable System Clock

The system clock is the official time keeper for the entire ADCS. Advantages to using a central clock instead of the computer time is that the speed of elapsed time can be modified at run-time during simulations to either compress the time to complete the simulation or slow down the simulation to inspect a certain event.

Quaternion Multiplicative Corrections

Quaternions that quantify a system's position contain 4 values that are commonly tracked and controlled separately. This produces a lot of errors since the 4 values are co-dependent and altering one modifies the rest as well. Breaking the values apart also creates a disconnect between the values and the physical position they help define. Use of the quaternion multiplicative correction technique maintains the integrity of the values and their relation to the physical position of the system. Usages of quaternions in most estimators and controllers require normalization of the state to a unit vector after each calculation. In this implementation, the only time normalization that should ever be required is to correct for floating point error accumulation.

Quaternion Scaling

Unlike body rates that can be linearly scaled, the 4 quaternion values are a sinusoidal value where multiple values can represent the same attitude (i.e. 0, 360 degrees). The common method of scaling by the raw values can have unexpected results. Due to the trig of small angle interacting with quaternions that represent small deviations in this manner introduces a little error, but the linear assumption creates larger errors as the angle increases. All quaternion scaling in this library is performed or the angle that the quaternion represents. With this method, we can maintain the linear scaling affect that is desired while maintaining the integrity of the quaternion representation.

Run-time interface

Through the use of a python twisted daemon process an restful API interface is available to query the state of the system in the middle of a run. This creates the ability to display meaningful representations of the system and increase insight into the system's dynamic behavior instead of relying on batch post processing.

Sample video

Concurrent Estimation/Control Algorithms

When running a comparative analysis between different types of estimators or different types of controllers, common methods are to re-run simulations for each variation and compare the results. The library allows for configurations such as providing the same measurement values to both a PID and SMO estimator to compare their performance.

Quaternion Decomposition

With spin-stabilized satellites the only part of the attitude quaternion that requires control is the one that quantifies the "wobble". Since the 4 values are co-dependent as mentioned above just modifying 2 of the values results in a corrupted position representation.

Modular Design

This library is designed to have interchangeable components with predefined and consistent interfaces and roles by allowing for the inclusion of additional estimation and control techniques. In the case of estimation, an Extended Kalman Filter (EKF) class can be added by creating a new class in Estimator.py that has contains the common properties ($\bs{\hat{x}}$, $\bs{x}_e$..) and common methods ($\bs{\hat{x}}$ = update($\bs{x}$)).

Portable Design

The portable design dictates that with the defined interfaces between modules, the observer based control methods have no knowledge of what is producing the sensor readings and what is accepting moment commands. This enforces consistency in the behavior of the system whether it's hooked up to an in-memory model of a satellite, NASA MMS TableSat 1A, or any future spin spin-stabilized platform. The only code change that may be required (Sensor.py and Actuator.py) is in applying the system to a new platform that contains new types of sensors and actuators.

Python

Since control systems professionals regularly work in a Numerical Simulation Software environment (like Matlab Simulink or Octave) and when it comes to implementation the logic is generally converted to a more standard language by software engineers there becomes a disconnect between the planning and implementation of a controller. By using a language like python, the code can be written in an expressive manner so that a control systems professional only needs a little programming experience to modify the code. It also keeps the controller in a language than could be reviewed for optimization by a software engineer and applied to the actual system without requiring a conversion to an entirely different system or language.

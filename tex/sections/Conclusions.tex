\chapter{CONCLUSIONS}
\label{chap:Conclusions}

The two main goals for this thesis are to:

\begin{itemize}
\item create a software application that runs the TableSat 1A attitude determination and control system (ADCS), but can be applicable to the wider class of spin-stabilized satellites.  The software should incorporate tools that provide the researcher opportunities to gain greater insight into the system's dynamics and interdependencies.
\item design a gyroless ADCS that is capable of maintaining a 3 rpm rotation while detecting and eliminating boom oscillations and satellite nutation.  The ADCS is to validate it's analytical operation against the experimental UNH NASA MMS TableSat 1A.
\end{itemize}

This thesis successfully covers the first goal of a software system to run the TableSat 1A's ADCS.  Though multiple iterations, the final TSatPy is a robust analytical and experimental application through which to assess the performance of control systems of spin-stabilized satellites.  The system handles intentional and unexpected variations in time steps caused by situations such as dropped packets in experimental testing or user manipulation of the system clock in simulations.  Through the system's adaptive step algorithms, all aspects of the estimators and controllers need not be preconfigured through linearization but can adjust their scale depending on the elapsed time since they were last updated.

The quaternion attitude parametrization plays a key role in the success of the ADCS as its numerical stability through rotations avoids issues such as gimbal lock for Euler angles and gracefully handled edge cases where control efforts switch direction upon passing the antipodal point, and attitude measurements pass through $360^o$ back to $0^o$.  Using the quaternion multiplicative method for all quaternion transformations ensures the integrity of the quaternion measurements rather than the traditional state difference and normalization process.  When adjusting the quaternion state, the theta multiplication and quaternion vector balancing preserves the integrity of the rotational quaternion while also providing a single gain to scale the attitude linearly with the angular displacement rather than linearly with the cosine of the half-angle.  The derivation of the quaternion decomposition, Equation (\ref{eqn:quaternion_decomposition_derivation}), also enabled a 5-DOF control system leaving the yaw attitude unrestricted.

The fourth version of the software incorporates some advanced visualizations bringing together information from various sources such as estimators, controllers, and actuators into a singe representation of the ADCS system that gives the researcher feedback on how the separate algorithms work together to create the whole system. (See http://vimeo.com/user11804289/videos for more samples)

Additionally the TSatPy system is design to be modular.  Multiple estimation and control techniques can run under their associated master instances.  This thesis demonstrated the concurrent testing of a PID and Sliding Mode Observer which illuminated subtle differences in their operation that would not likely be seen if tested separately.  The modular design extends to the entire system itself as it is written in a widely used and relatively easy to learn programming language so that future researcher can expand on the current foundation by adding more estimation or control techniques, incorporating more inspection tools, or modifying the ingress and output points to control other experimental platforms.

The second goal of designing an ADCS for TableSat 1A to maintain spin and correct for boom oscillation and s/c nutation is partially complete.  An analytical model of an observer-based controller of a Sliding Mode Observer with a very effective Sliding Mode Controller was developed and tested to control the 5-DOF problem.  A gradient descent algorithm (Appendix \ref{code:TSatPySamples/GradientDescent.py}) was created for the purpose of optimizing the gains to minimize the control effort and attitude error measurement.   Additional tuning of the model is possible, but would benefit from initially porting the tPlot library from the NSS format to python in better visualize how all the gain adjustments affect the overall performance.

For the experimental verification, the ADCS was successful at establishing a steady spin rate of 3 rpm on the TableSat using the two rotational fans for controllers.  The nutation control and related boom dynamics control goals are incomplete.  In this implementation of TableSat, the two single direction nutation fans are unable to produce the enough thrust fast enough to have any noticeable effect on damping any nutation or boom oscillation.  CO2 canisters were added experimentally to attempted to increase thrust, but force provided is too large and has a very limited capacity to make it a cost effective option

For attitude measurements the coarse sun sensors provides a fairly reliable measure of yaw such that the rate of change allows for an acceptable gyroless rate control.  Through a series of calibration techniques the triple axis magnetometer was successfully utilized to establish a method of comparing the current TAM readings to a calibration set from the same yaw measure and estimating the degree of nutation.  While on a large scale with many passes and densely sampled magnetic fields an initial collection of seemingly indistinguishable data points were smoothed out to establish predictable paths (Figure \ref{fig:TAMSignal}).  That success however still did not allow for the adequate nutation detection as it took longer to filter out the noise from the TAM voltage measurements as it did for the TableSat's nutation to naturally die out due to friction in the mount point.

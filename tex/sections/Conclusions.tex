\chapter{CONCLUSIONS AND FUTURE WORK}
\label{chap:Conclusions}

\section{Conclusions}

The two main goals for this thesis are to:

\begin{itemize}
\item Create a software application that enables the implementation of the TableSat IA attitude determination and control system (ADCS) but can be applicable to a wider class of spin-stabilized satellites.  The software should incorporate tools that provide the researcher opportunities to gain greater insight into the system's dynamics and interdependencies.
\item Design a gyroless ADCS that is capable of maintaining a 3 rpm rotation while detecting and eliminating boom oscillations and satellite nutation.  The ADCS is to validate its analytical operation against the experimental UNH NASA MMS TableSat IA.
\end{itemize}

This thesis successfully achieves the first goal.  Though multiple iterations, the final TSatPy is a robust analytical and experimental application through which to assess the performance of control systems of spin-stabilized satellites.  The system handles intentional and unexpected variations in time steps caused by situations, such as dropped packets in experimental testing or user manipulation of the system clock in simulations.  Through the system's adaptive step algorithms, all aspects of the estimators and controllers calculate elapsed time at each step, and as such are not preconfigured for a specific time step size.

The quaternion attitude parametrization plays a key role in the success of the ADCS as its numerical stability through rotations avoids issues such as gimbal lock for Euler angles.  It also gracefully handles edge cases where control efforts switch direction upon passing the antipodal point and when attitude measurements pass through $360^o$ rotation to return to $0^o$.  Using the quaternion multiplicative correction method for all quaternion transformations ensures the integrity of the quaternion measurements and does not rely on the traditional state difference and normalization process.  When adjusting the quaternion state, the theta multiplication and quaternion vector balancing preserves the integrity of the rotational quaternion while also providing a single gain to linearly scale the attitude along with the angular displacement, (rather than linearly with the cosine of the half-angle).  The derivation of the quaternion decomposition, Equation (\ref{eqn:quaternion_decomposition_derivation}), also enables a 5-DOF control system leaving the yaw attitude unrestricted.

The fourth version of the software incorporates some advanced visualizations bringing together information from various sources such as estimators, controllers, and actuators into a single representation of the ADCS system that gives the researcher feedback as to how the separate algorithms work together to create the whole system. (See http://vimeo.com/user11804289/videos for samples)

Additionally the TSatPy system is designed to be modular.  Multiple state estimation and control techniques can be implemented under their associated master instances.  This thesis demonstrates, for example, the concurrent testing of a PID state estimator and a Sliding Mode Observer, where it illuminates subtle differences in their operation that would not likely be seen if tested separately.  The modular design extends to the entire software package as a whole as well.  It is written in a widely used and relatively easy to learn programming language so that future researcher can expand on the current foundation by adding more estimation or control techniques, incorporating more inspection tools, or modifying the ingress and output points to control other experimental platforms.

The second goal of designing an ADCS for TableSat IA to maintain spin and correct for boom oscillation and spacecraft nutation is partially complete.  An analytical model of an observer-based controller of a Sliding Mode Observer with a very effective Sliding Mode Controller is developed and tested to control the 5-DOF problem.  A gradient descent algorithm (Appendix \ref{code:TSatPySamples/GradientDescent.py}) is created for the purpose of optimizing the gains to minimize the control effort and attitude error measurement.  Additional tuning of the model is possible but would benefit from initially porting the tPlot library from the NSS format to python to better visualize how all the gain adjustments affect the overall performance.

For experimental verification, the ADCS is successful at establishing a steady spin rate of 3 rpm on the TableSat using the two rotational fans as actuators.  The nutation control and related boom dynamics control goals are not yet complete and are left for future work .  In this implementation of TableSat, the two single direction nutation fans are unable to produce enough thrust or provide thrust quickly enough to have any noticeable effect on damping of nutation or boom oscillation.  $\text{CO}_2$ canisters are added to attempted to increase thrust, but the force provided is too large and has a very limited capacity to make it a cost-effective option.

For attitude measurements the coarse sun sensors provides a fairly reliable measure of yaw such that the rate of change allows for an acceptable gyroless rate control.  Through a series of calibration techniques the three-axis magnetometer is successfully utilized to establish a method of comparing the current TAM readings to a calibration set from the same yaw measure and estimating the degree of nutation.  While on a large scale with many passes and densely sampled magnetic fields an initial collection of seemingly indistinguishable data points are smoothed out to establish predictable paths (Figure \ref{fig:TAMSignal}).  That success, however, still does not allow for adequate nutation detection, as it requires more time to filter out the noise from the TAM voltage output measurements as it does for the TableSat's nutation to naturally dampen as a result of friction at the mount point.

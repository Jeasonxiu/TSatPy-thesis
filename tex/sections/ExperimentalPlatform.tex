
\chapter{Experimental Platform}
\label{chap:Experimental Platform}

\section{Communication With TableSat}
\label{sec:CommunicationWithTableSat}

Communication between the controller and the experimental TableSat is
negotiatied over a User Datagram Protocol (UDP) socket.  Both the
controller and TableSat read and write over port 9877.  UDP is used for
a stateless connection more generally used for pushing data over a
large number of connections.  The other common communication protocol
is Transmission Control Protocol (TCP) where a session is established
betwen server and client and succesful transferral of data is
acknowledged by the recipient.

The advantage to using UDP for the UNH TableSat 1 is that the sattelite
side code could be largely based off of Mellissa Vess' work on her
TableSat thesis where the control loop was implemented onboard, and
the UDP connection was used to infrequently poll for data or modify
sensor callibration values.  The UDP connection has the slight advantage
of being able to transmit and recieve packets without needing to wait for
acknowledgements.  This could save a small fraction of time, but on modern
hardware and with the small bandwidth usage this advantage is negligable.

Disadvantages to implementing the control interface through UDP include
the possibility of packet loss where the sender submits a packet, but
it gets corrupted or dropped.  Since UDP is a stateless connection, the
sender doesn't wait for an acknowledgement which in this case would not come.
This issue is handled by the communications module.  If the control loop rate
on both updating the actuator voltages and polling sensor data is fast enough,
a dropped packet would not matter much since a new updated set of values
would be soon to follow.  Issues could be caused if the completion of a
code loop was dependent on a packet coming in.  If dropped, the control
loop could block until the next packet is recieved.  This was one issue
encountered in the project version [sec:ControlLoopinSimulink].


\subsection{UDP Packet}

The packet structure used for this project is the same as used by Mellissa Vess.
Each UDP packet is comprised of a packet header and data payload.  The packet
header format remains constant for all messages containing five octets of data.

\begin{center}
    \begin{tabular}{| l | l |}
    \hline
    Header Octet & Description \\ \hline
    h1 & Message Number (0 to 255) \\ \hline
    h2 & Flags (0 - 255) \\ \hline
    h3 & Message Size (0 - 255) \\ \hline
    h4 & Message Size (0 - 255) \\ \hline
    h5 & 0 \\ \hline
    \end{tabular}
\end{center}

The first octet (h1) contains the message number that matches to a predefined list
of messages known by both the sender and recipient.  This is used to specify
how the data in the payload is to be used.

The second octet (h2) is reserved for flags.  For some messages, flags can be
set for additional data.  These are not used in the final implementation of
UNH TableSat 1A.

The third (h3) and fourth (h4) header octets define the size of the data's payload, so when reading
data in from a buffer, the header can inform the recipient how many bytes need
to get read in from the buffer in order to get to the end of the packet.  For
payloads less than 256 bytes only the fourth header byte is needed.  For
payloads larger than 255 bytes the following formula is used to specify the
message size headers.

\begin{align}
h4 &= mod(size, 256) \\
h3 &= floor(size / 256)
\end{align}


\subsection{Messages}

The only meta data provided along with the payload beyond the data's size is
an 8 bit message number.  Both UNH TableSat 1A and the control station have
identical message list definitions.

\begin{center}
    \begin{tabular}{| l | l | l | l |}
    \hline
    Message Id & Payload Size & Data Type & Message Description \\ \hline
    2 & 1 octet & unsigned int & Set run mode \\ \hline
    4 & 1 octet & unsigned int & Set run mode \\ \hline
    18 & 4 x (8 octets) & float & Set fan speed \\ \hline
    19 & 1 octet & unsigned int & Set log record mode \\ \hline
    20 & 1 octet & unsigned int & Request sensor reading \\ \hline
    22 & 1 octet & unsigned int & End of sensor log \\ \hline
    23 & 8 octets & float & Request sensor log data \\ \hline
    33 & 8 octets & float & Set log sample rate \\ \hline
    63 & 15 x (8 octets) & float & Sensor readings \\ \hline
    64 & 16 x (8 octets) & float & Sensor log entry \\ \hline
    65 & 8 octets & float & Sensor log size \\ \hline
    104 & 1 octet & unsigned int & Ack run mode \\ \hline
    118 & 1 octet & unsigned int & Ack fan volt \\ \hline
    119 & 1 octet & unsigned int & Ack sensor log run mode \\ \hline
    133 & 1 octet & unsigned int & Ack log sample rate \\ \hline
    \end{tabular}
\end{center}


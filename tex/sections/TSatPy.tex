
\chapter{TSatPy}
\label{chap:TSatPy}

While the NSS versions discussed in the previous chapter provide a level of insight and control into a control system, the NSS environment is not a natural fit for that level of programming and presents issues that take a significant effort to work around.  In addition to issues with diagnosing problems, the NSS controller code targets a fairly narrow swath of users that are comfortable with using and editing it.  Because of this, the final version is written in Python, a common cross-platform and open source programming language.

Python is quoted as coming ``batteries included'' which is a statement to the number of common libraries that come standard with the language including a \verb|socket| library for sending and receiving User Datagram Protocol (UDP) packets used by TableSat.  Python is also known as an expressive language where what a section of code does is generally easily discernible from a ``quick read''.

Python includes a variety of unit test frameworks for ensuring consistency throughout code revisions, and gives the ability to write self-documenting code.  Self-documenting code is such that if the user has an instance of a class stored in a variable \verb|v| but is unsure what can be done with the class, typing ``\verb|help(v)|'' provides documentation of the various methods available and how they are used.

Section \ref{sec:KeyCharacteristics} covers some of the notable characteristics about the TSatPy implementation of the ADCS, and Section \ref{sec:Module Design} describes the different types of classes that are written for the program and what their role is.

\section{Key Characteristics}
\label{sec:KeyCharacteristics}

\subsection{Adaptive Step Algorithms}

All time dependent calculations vary parameters at run-time depending on the last time it was executed. For example, the integral component of a PID controller with an error measure of $+0.2$ accumulates an error of $+0.02$ for the first $\Delta t_k = 0.1$ sec but only accumulates an additional $+0.016$ for the next time step of $\Delta t_{k+1} = 0.08$ sec.  This functionality avoids errors encountered when a system is linearized about an assumed time step.  However, the actual time step varies which causes the controller gains to be overly aggressive or conservative.

\subsection{Variable System Clock}

The system clock is the official time keeper for the entire ADCS. Advantages to using a central clock instead of that from the computer is that the rate of elapsed time can be modified at run-time during simulations to either compress the time to complete the simulation or slow the simulation to inspect a transient event. (See Section \ref{subsec:IntegralEstimator})

\subsection{Quaternion Multiplicative Corrections}

Quaternions that quantify a system's position contain 4 values that through common control systems implementations are tracked and controlled separately. This approach ignores the restriction that rotational quaternions maintain a unit norm while the adjustments are made and then scales the resulting quaternion back to a unit norm compromising the predictability and stability of the control system.  Using object oriented programming to track scalar and vector components ensures that controllers can be designed that make use of the quaternion multiplicative correction technique which maintains the integrity of the values and their relation to the physical attitude of the system. (See Section \ref{subsec:QuaternionAttitude})

\subsection{Theta Multiplier with Quaternion Vector Balancing}

Unlike body rates that can be linearly scaled, the 4 quaternion values are sinusoidal values.  Here, various sets of values can represent the same attitude (i.e. 0, 360 degrees). Scaling the sinusoidal values directly produces inconsistent adjustment rates. All quaternion scaling in the TSatPy software adheres to the following principles
\begin{enumerate}
  \item Adhere to the unit rotational quaternion restriction
  \item Do not modify the direction of the Euler axis $\bs{\hat{e}}$
  \item Scale $\theta$ (not $q_0$) so a $4^o$ error can be intuitively scaled to down to a $1^o$ adjustment with a selected gain value of 0.25.
\end{enumerate}
With this method, the desired linear scaling effect is achieved maintaining the integrity of the rotational quaternion representation. (See Section \ref{subsec:ThetaMultiplierWithQuaternionVectorBalancing})

\subsection{Run-time Interface}

The fourth version of the NSS application included the development of the tPlot library for dynamic and real-time 3D modeling of a system's state.  That same functionality is desired in the TSatPy implementation and initial work is completed on a Python ``twisted'' daemon with a RESTful API interface that coordinates running of the ADCS, communication with TableSat, and user access to ``live'' data.  (See Section \ref{sec:ObjectOrientedNSSControlSystem})

\subsection{Concurrent Estimation/Control Algorithms}

When running a comparative analysis between different types of estimators or different types of controllers, common methods are to re-run simulations for each variation and compare the results. The library allows for configurations such as providing the same measurement values to both a PID and SMO estimator to compare their performance. (See Section \ref{sec:ComparativeAnalysysofPIDandSMOEstimators})

\subsection{Quaternion Decomposition}

With spin-stabilized satellites, five degrees of freedom are available to compare with a fixed desired value.  The quaternion decomposition method allows the separation of the yaw motion from the remaining states, enabling the 5-DOF control of constant body rates and removing nutation. (See Section \ref{subsec:SpinStabilizedControl})

\subsection{Modular Design}

This library is designed to have interchangeable components with predefined and consistent interfaces and roles by allowing for the inclusion of additional estimation and control techniques. In the case of estimation, an Extended Kalman Filter (EKF) class can be added by creating a new class in Estimator.py that contains the common properties ($\bs{\hat{x}}$, $\bs{x}_e$..) and common methods ($\bs{\hat{x}}$ = update($\bs{x}$)).

\subsection{Portable Design}

Portable design dictates that with the defined interfaces between modules, the observer-based control methods have no knowledge of what is producing the sensor readings and what is accepting moment commands. This enforces consistency in the behavior of the system whether it is linked to an ``in-memory'' model of a satellite, NASA MMS TableSat IA, or any future spin spin-stabilized platform.

\subsection{Python}

There can be a disconnect between the control design developed using a Numerical Simulation Software environment (such as MATLAB/Simulink or Octave) and that programmed using a more standard language. By using a language like python, the code can be written in an expressive manner so that a control systems professional needs little programming experience to modify the code appropriately. It also keeps the controller in a language that could be reviewed for optimization by a software engineer and applied directly to the experimental platform without requiring a conversion to an entirely different system or language.


\section{Module Design}
\label{sec:Module Design}

Below is an outline of the separate modules contained in the TSatPy library and their associated roles.

\subsection{Actuators}
\label{subsec:actuators}

input: Desired moment\\
output: Actual moment\\

This module defines the output of the controller. The actuator instance on initialization is supplied with the physical arrangement of the actuators. During run-time the actuator module first accepts a $\Re^3$ moment command from the controller.  It then compares what it is being asked to do with what it can actually provide.  That true moment is then available for use with estimator state propagation and for conversion to actuator voltage commands. (See Section \ref{sec:ActuatorConfiguration})

\subsection{ADCS}
\label{subsec:ADCS}

input: Configuration file\\
output: ADCS model\\

This module is intended as the main parent object.  As an input it reads a json configuration file that defines all the components and parameters required for a specific observer-based controller.


\subsection{Clock}
\label{subsec:Clock}

input: None\\
output: Elapsed time\\

This module establishes the authoritative source for elapsed time and is continually referenced by any time-dependent logic such as integral or derivative based parameters. The metronome class by default tracks time after its initialization.  When running simulations, its speed can be dynamically altered to either run faster for long-term simulations or slower to enable more detailed inspection of an event.


\subsection{Comm}
\label{subsec:Comm}

From Controller to Satellite\\
input: Voltage setting determined by the actuator\\
output: UDP packet to the system or in-memory model\\
From Satellite to Controller\\
input: Sensor voltages from the system or in-memory model\\
output: Sensor data to submit to sensor class for conversion to a state\\

This module handles the interface between the control system and either the physical system or an analytical model. With the physical system, a UDP socket is opened and either listens for packets from the sensors or submits voltage command packets to set actuator voltages. During a simulation, the module submits and receives the voltage command messages to an ``in-memory model'' of the system that contains the system dynamics and can return simulated sensor voltages based on the behavior.


\subsection{Controller}
\label{subsec:Controller}

input: Estimated state ($\bs{\hat{x}}$)\\
output: Moment couples ($\bs{M}$)\\

This module contains the algorithms that compare the estimated system behavior with desired behavior and determines what appropriate moment commands are required for the desired operation.

The master controller instance governs the interface between the incoming estimator state and the outgoing actuator moments. The master controller can run multiple control algorithms simultaneously which can be individually tuned to different types of system behaviors such as one that can accommodate for large errors and one for soft corrections (steady state).


\subsection{Estimator}
\label{subsec:Estimator}

input: Measured state ($\bs{x}$)\\
output: Estimated state ($\bs{\hat{x}}$)\\

This module contains the algorithm that uses the measured state from the sensor model, which is likely based on noisy sensor data, to attenuate the effects of noise to create an accurate representation of the true state of the system.

\subsection{Sensor}
\label{subsec:Sensor}

input: Voltages ($\bs{V}$)\\
output: Measured state ($\bs{x}$)\\

This module receives raw voltage measurements from either a simulated ``truth model'' or from the Comm module polling sensor voltage data from the experimental TableSat.  Each class represents a different sensor type (coarse sun sensor, magnetometer, gyroscope, ...) and contains the logic to convert sensor readings to state measurement (i.e. quaternion and body rates).


\subsection{Service}
\label{subsec:Service}

input: Run configuration\\
output: Twisted daemon\\

Coupled with the ADCS module the Service module is intended to create the daemon that manages the interactions between all the modules including sockets to TableSat.


\subsection{StateOperator}
\label{subsec:StateOperator}

input: An instance from the State module\\
output: A modified State instance or the conversion to a new State instance\\

This module contains classes that are designed to modify instances from the State module such as \verb|BodyRate| or \verb|Quaternion| or convert between them. For example an error quaternion can be scaled larger or smaller by the QuaternionGain class.

\begin{table}[H]
  \centering
  \begin{tabular}{l|p{0.5\linewidth}}
    Class &  What it does \\ \hline
    BodyRateGain & Scale a BodyRate instance\\
    QuaternionGain & Scale a Quaternion instance\\
    StateGain &  Scale the State instance (Wrapper for QuaternionGain, BodyRateGain)\\
    QuaternionSaturation & Saturate a Quaternion\\
    BodyRateSaturation & Saturate a BodyRate\\
    StateSaturation  & Saturate a State (Wrapper for QuaternionSaturation, BodyRateSaturation)\\
    BodyRateToMoment & Convert a BodyRate to a Moment\\
    QuaternionToMoment & Convert a Quaternion to a Moment\\
    StateToMoment  & Convert a State to a Moment (Wrapper for QuaternionToMoment, BodyRateToMoment)\\
  \end{tabular}
  \caption{State Operators}
  \label{tbl:StateOperators}
\end{table}



\subsection{State}
\label{subsec:State}

This module contains classes that are used to quantify the states of the system.

\begin{table}[H]
  \centering
  \begin{tabular}{l|p{0.5\linewidth}}
State  & Class \\ \hline
Quaternion & Attitude \\
BodyRate & Body-fixed angular velocities \\
QuaternionError & Difference in attitudes \\
QuaternionDynamics & How body rates change the attitude \\
EulerMomentEquations & How moments change the body rates  \\
State & Position and velocity \\
StateError & Differences in position and velocity \\
Plant & In-memory model of the satellite \\
Moment & Applied torques \\
  \end{tabular}
  \caption{State Measurements}
  \label{tbl:State}
\end{table}



\chapter{Background}
\label{chap:Background}

The National Aeronautics and Space Administration (NASA) is planning to launch the Magnetospheric Multiscale (MMS) Mission into orbit around the Earth in order to study reconnection of the magnetic fields.  The effects of the phenomenon can be clearly seen in disturbances to spacecraft, GPS signals, and even inducing current in the power grid to the point of causing outages.  Although the effects are redily observed the internal dynamics and modelling is only theorized.  Magnetic field reconnection traces back to coronal mass ejections (CME) from the sun.  These CMEs contain their own magnetic field properties that when collide with the fields surrounding the Earth can cause opposing field lines to cross.  The crossing and separation is called reconnection.  It's occurance can open up portals connecting plasma from the interstellar medium to that of the Earth's magnetosphere which can produce a release of a large amount of energy.  A reconnection event starts on the dayside magnetopause of the Earth and
rapidly folds over to magnetotail.  The diffusion region of the event can travel at about \todo{lookup speed} and has a thickness of \todo{lookup thickness} which can pass orbiting satellites in a fraction of a second.  Theories about the internal structure of the diffusion region \todo{reference active theories} have been difficult to test because of the speed of the event.  The general approach to getting a full view of the area surrounding a satellite is to mount a sensor facing out from the satellite and set the satellite to be spin stabilized so that through the turn a series of observations can be made and then stitched together to form a full field of view.  In order for this approach to be partially successful the satellite would have to be spinning at about 600 rpm to get a single full view image, which for a moderate sized satellite would introduce a restrictive amount of rotational dynamics coumpounded by the fact that each MMS satellite has 4 Spin-plane Double Probe (SDP) booms \todo{work in
reference to UNH R. B. Torbert for FIELDS investigation} extending about 50 meters out from the spacecraft (s/c) body and two Axial Double Probe (ADP) booms extending 10 meters out along the rotational axis.  The potential 600 rpm speed would translate into a SDP tip velocity around 3,000 meters per second.
The approach taken by the MMS team is to mount a series of instruments around the s/c to have a full view at all times.  This in conjunction with the rate of measurements each instrument can take will provide a much richer set of data for the mission scientists to analyze.
